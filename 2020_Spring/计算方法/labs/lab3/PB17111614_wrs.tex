\documentclass{article}
\usepackage{graphicx}
\usepackage{fontspec}
\setmainfont{Microsoft YaHei}
\usepackage{geometry}
\setlength{\parindent}{0pt}   %控制缩进
\usepackage{ctex}
\usepackage{algorithm}  
\usepackage{algorithmicx}  
\usepackage{algpseudocode}  
\usepackage{amsmath}  
\title{计算方法 Lab3}
\author{王嵘晟 \quad PB1711614}
\date{}
\begin{document}
	\maketitle
	\Large\section{实验结果:}
	对于$f(x)=2x^{4}+24x^{3}+61x^{2}-16x+1=0$
	\large\subsection*{(1).牛顿迭代法:}
	当取初值$x_{0}=0$时:\\
	\makeatletter\def\@captype{table}\makeatother
	\large
	\caption{Newton迭代结果1}  
	\begin{center}  
		\begin{tabular}{|c| c | c |}  
			\hline  
			迭代步数$k$ & $x_{k}$ & $f(x_{k})$  \cr \hline  
			$0$ & $0.0000000000E+000$ & $1.0000000000E+000$ \cr \hline
			$1$ & $6.2500000000E-002$ & $2.4417114258E-001$ \cr \hline
			$2$ & $9.2675144823E-002$ & $6.0357821710E-002$ \cr \hline
			$3$ & $1.0750916023E-001$ & $1.4994760152E-002$ \cr \hline
			$4$ & $1.1485323376E-001$ & $3.7248898748E-003$ \cr \hline
			$5$ & $1.1848368152E-001$ & $9.1626064336E-004$ \cr \hline
			$6$ & $1.2024260677E-001$ & $2.1577268802E-004$ \cr \hline
			$7$ & $1.2102581790E-001$ & $4.2847681852E-005$ \cr \hline
			$8$ & $1.2128383271E-001$ & $4.6530959359E-006$ \cr \hline
			$9$ & $1.2131962667E-001$ & $8.9569062833E-008$ \cr \hline
			$10$ & $1.2132034327E-001$ & $3.5900726836E-011$ \cr \hline
		\end{tabular}  
	\end{center}
	当取初值$x_{0}=3$时:\\
	\makeatletter\def\@captype{table}\makeatother
	\large
	\caption{Newton迭代结果2}  
	\begin{center}  
		\begin{tabular}{|c| c | c |}  
			\hline  
			迭代步数$k$ & $x_{k}$ & $f(x_{k})$  \cr \hline  
			$0$ & $3.0000000000E+000$ & $1.3120000000E+003$ \cr \hline
			$1$ & $1.9192751236E+000$ & $3.9180729077E+002$ \cr \hline
			$2$ & $1.1936133228E+000$ & $1.1368262566E+002$ \cr \hline
			$3$ & $7.3112145458E-001$ & $3.1859876182E+001$ \cr \hline
			$4$ & $4.5362080609E-001$ & $8.6190504416E+000$ \cr \hline
			$5$ & $2.9663693142E-001$ & $2.2633471161E+000$ \cr \hline
			$6$ & $2.1197535112E-001$ & $5.8197426653E-001$ \cr \hline
			$7$ & $1.6779403531E-001$ & $1.4770709460E-001$ \cr \hline
			$8$ & $1.4519438879E-001$ & $3.7206334228E-002$ \cr \hline
			$9$ & $1.3376760731E-001$ & $9.3253679860E-003$ \cr \hline
			$10$ & $1.2803649704E-001$ & $2.3222638207E-003$ \cr \hline
			$11$ & $1.2519603327E-001$ & $5.6755417123E-004$ \cr \hline
			$12$ & $1.2383872242E-001$ & $1.2927049695E-004$ \cr \hline
			$13$ & $1.2327102895E-001$ & $2.2587102744E-005$ \cr \hline
			$14$ & $1.2311856261E-001$ & $1.6284754711E-006$ \cr \hline
			$15$ & $1.2310571808E-001$ & $1.1556329893E-008$ \cr \hline
			$16$ & $1.2310562562E-001$ & $5.9885429948E-013$ \cr \hline
		\end{tabular}  
	\end{center}
	\large\subsection*{(2).弦截法:}
	取初值$x_{0}=0,x_{1}=0.5$时:\\
	\makeatletter\def\@captype{table}\makeatother
	\large
	\caption{弦截法迭代结果1}  
	\begin{center}  
		\begin{tabular}{|c| c | c |}  
			\hline  
			迭代步数$k$ & $x_{k}$ & $f(x_{k})$  \cr \hline  
			$0$ & $0.0000000000E+000$ & $1.0000000000E+000$ \cr \hline
			$1$ & $5.0000000000E-001$ & $1.1375000000E+001$ \cr \hline
			$2$ & $-4.8192771084E-002$ & $1.9100839450E+000$ \cr \hline
			$3$ & $-1.5882177241E-001$ & $4.9849583298E+000$ \cr \hline
			$4$ & $2.0528956326E-002$ & $6.9745241532E-001$ \cr \hline
			$5$ & $4.9704099508E-002$ & $3.5839401507E-001$ \cr \hline
			$6$ & $8.0543024888E-002$ & $1.1965360731E-001$ \cr \hline
			$7$ & $9.5999095782E-002$ & $4.7582804260E-002$ \cr \hline
			$8$ & $1.0620354980E-001$ & $1.7777847602E-002$ \cr \hline
			$9$ & $1.1229022963E-001$ & $6.8102183582E-003$ \cr \hline
			$10$ & $1.1606968076E-001$ & $2.5795784215E-003$ \cr \hline
			$11$ & $1.1837415259E-001$ & $9.7415265691E-004$ \cr \hline
			$12$ & $1.1977247782E-001$ & $3.6072356011E-004$ \cr \hline
			$13$ & $1.2059475518E-001$ & $1.2741741436E-004$ \cr \hline
			$14$ & $1.2104383231E-001$ & $3.9878767168E-005$ \cr \hline
			$15$ & $1.2124841215E-001$ & $9.3453570276E-006$ \cr \hline
			$16$ & $1.2131102788E-001$ & $1.1695181286E-006$ \cr \hline
			$17$ & $1.2131998479E-001$ & $4.4816937383E-008$ \cr \hline
			$18$ & $1.2132034170E-001$ & $2.3240176450E-010$ \cr \hline
		\end{tabular}  
	\end{center}
	取初值$x_{0}=0.1,x_{1}=1.5$时:\\
	\makeatletter\def\@captype{table}\makeatother
	\large
	\caption{弦截法迭代结果2}  
	\begin{center}  
		\begin{tabular}{|c| c | c |}  
			\hline  
			迭代步数$k$ & $x_{k}$ & $f(x_{k})$  \cr \hline  
			$0$ & $1.0000000000E-001$ & $3.4200000000E-002$ \cr \hline
			$1$ & $1.5000000000E+000$ & $2.0537500000E+002$ \cr \hline
			$2$ & $9.9766826661E-002$ & $3.4920022729E-002$ \cr \hline
			$3$ & $9.9528703766E-002$ & $3.5662994682E-002$ \cr \hline
			$4$ & $1.1095871197E-001$ & $8.7722830757E-003$ \cr \hline
			$5$ & $1.1468740718E-001$ & $3.8969392942E-003$ \cr \hline
			$6$ & $1.1766781216E-001$ & $1.3876451955E-003$ \cr \hline
			$7$ & $1.1931598272E-001$ & $5.3099453171E-004$ \cr \hline
			$8$ & $1.2033760050E-001$ & $1.9023231922E-004$ \cr \hline
			$9$ & $1.2090792407E-001$ & $6.3397280115E-005$ \cr \hline
			$10$ & $1.2119299484E-001$ & $1.7038550552E-005$ \cr \hline
			$11$ & $1.2129776891E-001$ & $2.8550135170E-006$ \cr \hline
			$12$ & $1.2131885895E-001$ & $1.8556909953E-007$ \cr \hline
			$13$ & $1.2132032505E-001$ & $2.3119210990E-009$ \cr \hline
			$14$ & $1.2132034354E-001$ & $1.9196866319E-012$ \cr \hline
		\end{tabular}  
	\end{center}
	\section{算法分析:}
	使用C语言编写程序,分别实现了对于多项式方程的Newton迭代法和弦截法通用求解。\\
	对于牛顿迭代,使用公式为:
	$$x_{k+1}=x_{k}-\frac{f(x)}{f'(x)}$$
	对于弦截法,使用公式:
	$$x_{k+1}=x_{k}-f(x_{k})\frac{x_{k}-x_{k-1}}{f(x_{k})-f(x_{k-1})}$$
	经过多次迭代之后,直到误差小于$\epsilon$为止,即$|f(x_{k})|<1e-9$
	\section{结果分析:}
	四种求解方式的最终结果如下表:\\
	\makeatletter\def\@captype{table}\makeatother
	\large
	\caption{迭代计算结果}  
	\begin{center}  
		\begin{tabular}{|c|c|c|c|c|}  
			\hline  
			迭代方法 & 初值 & 迭代次数$k$ & $x_{k}$ & $f(x_{k})$ \cr \hline 
			Newton & 0 & $10$ & $1.2132034327E-001$ & $3.5900726836E-011$ \cr \hline
			Newton & 3 & $16$ & $1.2310562562E-001$ & $5.9885429948E-013$ \cr \hline
			弦截法 & 0,0.5 & $18$ & $1.2132034170E-001$ & $2.3240176450E-010$ \cr \hline
			弦截法 & 0.1,1.5 & $14$ & $1.2132034354E-001$ & $1.9196866319E-012$ \cr \hline
		\end{tabular}  
	\end{center}
	所以根据迭代计算结果可得:f(x)=0有两个根,分别在0.121附近和0.123附近
	\section{实验小结:}
	本次实验由于需要编写通用程序,相较以往的实验在代码编写上复杂度有所提升。使用Newton迭代和弦截法迭代时,使用不同的初值迭代到同一个误差范围下的迭代次数是不同的。相对来说Newton法迭代次数更少。选择好的初值可以减少迭代次数。
\end{document}