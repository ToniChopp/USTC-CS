\documentclass{article}
\usepackage{graphicx}
\usepackage{fontspec}
\setmainfont{Microsoft YaHei}
\usepackage{geometry}
\setlength{\parindent}{0pt}   %控制缩进
\usepackage{ctex}
\usepackage{algorithm}  
\usepackage{algorithmicx}  
\usepackage{algpseudocode}  
\usepackage{amsmath}  
\title{计算方法 Lab4}
\author{王嵘晟 \quad PB1711614}
\date{}
\begin{document}
	\maketitle
	\Large\section{实验结果:}
	对于线性方程组 $Ax=b$,其中\\
	\includegraphics*[scale = 0.35]{1.png}\\
	\large\subsection*{(1).Jacobi迭代:}
	当取初始迭代$x^{(0)}=\{0,0,...,0\}^{T}$,停止条件$||x^{(k+1)}-x^{(k)}||_{\infty}\le 10^{-5}$时:\\
	程序部分运行结果如图:\\
	\includegraphics*[scale = 0.5]{2.png}\\
	根据给定的初始条件与停止条件,对于该线性方程组一共迭代了333次最终得出结果,该方程组的精确解为:\\
	$$x=(10,18,24,28,30,30,28,24,18,10)^{T}$$
	\large\subsection*{(1).Gauss-Seidel迭代:}
	当取初始迭代$x^{(0)}=\{0,0,...,0\}^{T}$,停止条件$||x^{(k+1)}-x^{(k)}||_{\infty}\le 10^{-5}$时:\\
	程序部分运行结果如图:\\
	\includegraphics*[scale = 0.5]{3.png}\\
	根据给定的初始条件与停止条件,对于该线性方程组一共迭代了175次最终得出结果,该方程组的精确解为:\\
	$$x=(10,18,24,28,30,30,28,24,18,10)^{T}$$
	\section{算法分析:}
	本题分别使用C语言编写\ Jacobi\ 迭代和\ Gauss-Seidel\ 迭代的通用程序\\
	对于\ Jacobi\ 迭代,使用分量形式的迭代公式:\\
	\includegraphics*[scale = 0.5]{4.png}\\
	作为程序编写的大体逻辑框架,使用三层循环来求解。设矩阵阶数为N,迭代次数为M,时间复杂度$O(MN^{2})$\\
	对于\ Gauss-Seidel\ 迭代,同样使用分量形式的迭代公式:\\
	\includegraphics*[scale = 0.6]{4.png}\\
	作为程序编写的大体逻辑框架,同样使用三层循环来求解,只需对最内层循环稍加修改即可。设矩阵阶数为N,迭代次数为M,时间复杂度$O(MN^{2})$\\
	\section{结果分析:}
	两种迭代方式的最终结果:
	$$x=(10,18,24,28,30,30,28,24,18,10)^{T}$$
	结果是一样的,\ Jacobi\ 迭代的迭代次数为333次,而\ Gauss-Seidel\ 迭代的迭代次数只有175次,显然\ Gauss-Seidel\ 迭代的收敛速度更快。所以就运算而言,\ Gauss-Seidel\ 迭代更具有优势。
	\section{实验小结:}
	本次实验同样需要编写通用程序,不过程序代码框架较为简单,且两种迭代方式的代码只需要做轻微修改就可以实现。但对于迭代求解线性方程组的两种方法还是有了一定的深入认识。\ Gauss-Seidel\ 迭代运算优于\ Jacobi\ 迭代
\end{document}