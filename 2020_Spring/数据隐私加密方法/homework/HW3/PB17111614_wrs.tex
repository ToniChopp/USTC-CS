\documentclass{article}
\usepackage{graphicx}
\usepackage{fontspec}
\setmainfont{Microsoft YaHei}
\usepackage{geometry}
\setlength{\parindent}{0pt}   %控制缩进
\usepackage{ctex}
\usepackage{algorithm}  
\usepackage{algorithmicx}  
\usepackage{algpseudocode}  
\usepackage{amsmath}  
\usepackage{amsthm,amsmath,amssymb}
\usepackage{mathrsfs}
\title{数据隐私加密方法 HW3}
\author{王嵘晟 \quad PB1711614}
\date{}
\begin{document}
	\maketitle
	\section*{1.}
	"alpha"转化为二进制:0110000101101100011100000110100001100001\\
	"bravo"转化为二进制:0110001001110010011000010111011001101111\\
	"delta"转化为二进制:0110010001100101011011000111010001100001\\
	"gamma"转化为二进制:0110011101100001011011010110110101100001\\
	由此可判断出: $c_{1}$是由 "alpha" 加密而来, $c_{2}$由"bravo"加密而来,按位异或后求得k:
	$$k=1001100000010101101111000111111111100111‬$$
	\section*{2.}
	令 $m_{L}=\{0\}^{\lambda}$,$m_{R}=\{1\}^{\lambda}$,这样左边得到的c为k本身,右边得到的c为k按位取反。所以不可以互换
	\section*{3.}
	$\frac{1}{2^{\frac{\lambda}{2}}},\frac{1}{\lambda^{log\lambda}},\frac{1}{2^{(log\lambda)^2}},\frac{1}{2^{\sqrt{\lambda}}}$,这些函数与$\lambda^{c}$相乘,当$\lambda \rightarrow \infty$时,积趋近于0
	\section*{4.}
	\subsection*{a.}
	用$\mathcal{A}$来验证$\mathcal{L}_{prg-real}^{G}$时,当且仅当遍历到$s'$与$s$完全一致时,才会返回1。但验证$\mathcal{L}_{prg-rand}^{G}$时可能找不到,返回0。所以是可以区分的,非 negligible。
	\subsection*{b.}
	不违背,通过G生成的长度为$\lambda+l$的伪随机数依然是均匀分布中不可区分的,没有违背prg的定义。
	\section*{5.}
	\subsection*{a.}
	$p=101,q=73,n=p\times q=7373$\\
	$\Phi(n)=(p-1)\times(q-1)=7200$\\
	此时选取e使得$1<e<\Phi(n)$,且$e$与$\Phi(n)$互质。则e共有1919个,所以由$(n,e)$组成的公钥共有1919个
	\subsection*{b.}
	$c=M^{e}modN$所以当$M=2008,e=91,N=7373$时,密文$c=2008^{91}mod\ 7373=2957$
	\subsection*{c.}
	先计算私钥,$ed=1mod\ \Phi(n)$,所以$d=2661$\\
	解密:$M=2957^{2661}mod\ 7373=2008$
	\section*{6.}
	$\Phi(N)=(p-1)\times(q-1),N=p\times q$,所以$\Phi(N)=p\times q-p-q+1=N-(p+q)+1$。所以可得到方程组:
	$$p\times q = N$$
	$$p+q = p\times q-\Phi(N)+1$$
	整理得$p^2-(N-\Phi(N)+1)p+N=0$
	这个二次方程可以在多项式时间内求解。
\end{document}