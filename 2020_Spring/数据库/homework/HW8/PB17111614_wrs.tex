\documentclass{article}
\usepackage{graphicx}
\usepackage{fontspec}
\setmainfont{Microsoft YaHei}
\usepackage{geometry}
%\setlength{\parindent}{0pt}   %控制缩进
\usepackage{ctex}
\usepackage{algorithm}  
\usepackage{algorithmicx}  
\usepackage{algpseudocode}  
\usepackage{amsmath}  
\title{数据库 	HW8}
\author{王嵘晟 \quad PB1711614}
\date{}
\begin{document}
	\maketitle
	\section*{1.}
	以两个并发事务$T_{1},T_{2}$为例,多个并发事务时同理。当出现以下两种情况时,不可串行化:\\
	\\
	(1)事务$T_{1}$ 写某个数据对象 A ,$T_{2}$读或写 A;\\
	(2)\ $T_{1}$ 读某个数据对象 A ,$T_{2}$写 A;\\
	\par{设$T_{1},T_{2}$的潜在冲突公共对象为${A_{1},A_{2},...,A_{n}}$,其中$X=\{A_{1},...A_{j}\}$符合(1),$Y=\{A_{j+1},...,A_{n}\}$符合(2)。所以此时$T_{1}$需要用X锁$①$,而$T_{2}$用S锁或X锁$②$。假设$①$先执行,则$T_{1}$获得锁,$T_{2}$等待。由2PL,$T_{1}$在全部获得X与Y中的对象的锁后才会开始释放锁。如果此时对于$A_{i},1\le i\le n$,$T_{2}$获得了$A_{i}$的锁,则会发生死锁。否则只有当$T_{1}$全部执行完了对于X,Y的事务才会释放锁,让等待的$T_{2}$开始执行。这使得$T_{1},T_{2}$可以顺序串行执行,则$T_{1},T_{2}$的调度是冲突可串的,因而是可串化调度。对于先执行$②$同理可证。因此若并发事务遵循2PL,在不发生死锁的情况下,对这些事务的并发调度一定是可串行化的调度。}
	\section*{2.}
	假设两个事物$T_{1},T_{2}$,对于一个数据对象A,$T_{1}$取得了S锁,$T_{2}$取得了U锁,则$T_{1}$只读A,$T_{2}$拿到了U锁可以升级到X锁对A做写操作,这样有良好的并发性。而当$T_{1}$取得了A的U锁时,若此时$T_{2}$取得A的S锁,则会导致$T_{1}$永远没法升级到X锁,无法对A进行写操作。这在DBMS中是无效的,所以为了避免这种情况,DBMS设计了不对称的锁相容规则,旨在提高并发性且可以正常对数据库进行事务操作。
	\section*{3.}
	对于普通2PL会,例如:\\
	\begin{center}
		\begin{tabular}{|c | c | c|}
			t & $T_{1}$ & $T_{2}$ \cr \hline
			1& xL1(A) & \cr \hline
			2& A=A-100 &  \cr \hline
			3& xL1(B) & \cr \hline
			4& Write(A) & \cr \hline
			5 &xL1(B) & \cr \hline
			6 &Unlock(A) & \cr \hline
			7 & &sL2(A)  \cr \hline
			8 &B=B+100 & Read(A)  \cr \hline
			9 &Write(B) & \cr \hline
			10& Unlock(B) & \cr \hline
			11 &Rollback & \cr \hline
		\end{tabular}
	\end{center}
	满足2PL,但由于最后$T_{1}$\ Rollback,导致$T_{2}$对A的Read是脏读\\
	\par{但对于严格2PL,X锁会保持到事务结束,当$T_{1}$对事务A持有X锁时,$T_{2}$不能持有A的任何锁,所以此时不会出现脏读,强2PL同理不会出现脏读。}
	\section*{4.}
	对于给定的操作串,其优先图如下:\\
	\includegraphics*[scale = 0.16]{1.jpg}\\
	由于优先图中有环,所以是不可串。
	\section*{5.}
	\begin{center}
		\begin{tabular}{|c | c | c| c | c |}
			\hline
			t & $T_{1}$ & $T_{2}$ & $T_{3}$ & $T_{4}$\cr \hline
			1&  & & xL3(x) &  \cr \hline
			2&  & & write(x) & xL4(y) \cr \hline
			3& xL1(z) & & & write(y) \cr \hline
			4& write(z)& & xL3(a) & u4(y) \cr \hline
			5 & & & write(a) & \cr \hline
			6 & & & xL3(y) & \cr \hline
			7 & & & write(y) &  \cr \hline
			8 & & xL2(a) & write(x) & \cr \hline
			9 & & wait & write(y) & \cr \hline
			10& & wait & u3(x) & \cr \hline
			11 & xL1(a) & wait & u3(a) & \cr \hline
			12 & wait & write(a) & u3(y) & \cr \hline
			13 & wait & xL2(z) & write(y) & \cr \hline
			14 & xL1(z) & wait & & \cr \hline
			15 & wait & wait & & \cr \hline
			16 & wait  & wait & & \cr \hline
		\end{tabular}
	\end{center}
	这时根据2PL,$T_{1}$等待$T_{2}$持有的a的锁,而$T_{2}$等待$T_{1}$持有的z的锁,出现了死锁情况,所以不能用用2PL调度器生成并发调度。
\end{document}