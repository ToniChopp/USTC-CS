\documentclass{article}
\usepackage{graphicx}
\usepackage{fontspec}
\setmainfont{Microsoft YaHei}
\usepackage{geometry}
\setlength{\parindent}{0pt}
\usepackage{ctex}
\usepackage{algorithm}  
\usepackage{algorithmicx}  
\usepackage{algpseudocode}  
\usepackage{amsmath}  
\title{Database HW7}
\author{王嵘晟 \quad PB1711614}
\date{}
\begin{document}
	\maketitle
	\section*{1.什么是事务的 ACID 性质?请给出违背事务 ACID 性质的具体例子,每个性质举一个例子。}
	ACID性质是事务的四种性质的首字母缩写,分别是原子性、一致性、隔离性、持久性。\\
	原子性:事务是不可分的原子,其中的操作要么都做,要么都不
	做。例:银行转账,A转账200元给B,可以分为三个操作,检查A账户余额高于200、A账户余额减去200、B账户余额加上200。这三个操作要么都做完,要么全部不做回滚。\\
	一致性:事务的执行保证数据库从一个一致状态转到另一个一致
	状态。例:依然是上一个转账的例子,三个操作进行都结束了才会提交事务,如果在执行A账户余额减去200、B账户余额加上200之间系统崩溃中断了,A不会凭空损失200,因为事务最终没有提交,所有事务中所作的修改也不会保存到数据库中。\\
	隔离性:多个事务一起执行时相互独立。例:当A给B转账200元这个事务在执行中时,还有C给A转账100元这个事务同时执行,这两个事务互不干扰,在事务提交前不会实际改变A账户余额。\\
	持久性:事务一旦成功提交,就在数据库永久保存。例:转账这个事务在提交后,A账户余额减少200,B账户余额增加200已经完成且会永久保存,即使系统崩溃也不会改变。
	\section*{2.如果一个存储过程A内部调用了另一个存储过程B,此时A和B是否都可以使用事务编程并保证事务的 ACID 性质?请解释你的理由}
	不可以,存储过程中使用事务编程,要么事务提交,要么回滚。假设存储过程A中的事务编程需要存储过程B中事务提交后得到的一致状态结果,假设B中事务回滚,则A中事务的执行会受到影响。这时隔离性得不到保障,所以不可以
	\section*{3.}
	\subsection*{对于①:}
	使用Undo/Redo机制\\
	1.Undo列表:{T2,T3},Redo:{T1}\\
	2.Undo\\
	T3:E=50\\
	T2:D=40\\
	T2:C=30\\
	3.Redo\\
	T1:A=40\\
	T1:B=60\\
	T1:A=75\\
	4.Write\ log\\
	<Abort.T2>\\
	<Abort.T3>\\
	所以A=75,\ B=60,\ C=30,\ D=40,\ E=50,\ F=60,\ G=70
	\subsection*{对于②:}
	使用Undo/Redo机制\\
	1.Undo列表:{T3},Redo:{T1,T2}\\
	2.Undo\\
	T3:E=50\\
	3.Redo\\
	T1:A=40\\
	T1:B=60\\
	T1:A=75\\
	T2:C=50\\
	T2:D=80\\
	T2:D=65\\
	T2:C=75\\
	4.Write\ log\\
	<Abort.T3>\\
	所以A=75,\ B=60,\ C=75,\ D=65,\ E=50,\ F=60,\ G=70
	\subsection*{对于③:}
	使用Undo/Redo机制同时考虑检查点\\
	1.Undo列表:{T4},Redo列表为空\\
	2.Undo\\
	T4:G=70\\
	T4:F=60\\
	3.Wirte\ log\\
	<Abort T4>\\
	所以A=75,\ B=60,\ C=75,\ D=65,\ E=90,\ F=60,\ G=70
\end{document}