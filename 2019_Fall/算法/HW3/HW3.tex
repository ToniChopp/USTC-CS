\documentclass[UTF8]{ctexart}  
\title{HW3}
\author{王嵘晟 \quad PB1711614}
\date{}
\begin{document}
	\maketitle
	\section*{1.}
	\par{MAX-HEAPIFY算法中,每个孩子的子树的大小之多为$\frac{2n}{3}$,而调整A[i]、A[LEFT(i)]、和A[RIGHT(i)]的关系的时间代价为$\Theta$(1),所以MAX-HEAPIFY的运行时间为:}\\
	        $T(n)\le T(\frac{2n}{3})+\Theta(1)$
	\par{所以由主方法:a=1,b=$\frac{3}{2}$,f(n)=$\Theta$(1)=$\Theta(n^{0})$=$\Theta(n^{log_{\frac{3}{2}}1})$}
	\par{所以T(n)=$\Theta(n^{log_{\frac{3}{2}}1}lgn)$=$\Theta(n^{0}lgn)$=$\Theta(lgn)$}
	\par{MAX-HEAPIFY的时间复杂度是O(lgn)得证}\\
	\par{由于包含n个元素的堆的高度为$\lfloor lgn\rfloor$,高度为h的结点最多包含$\lceil\frac{n}{2^{h+1}}\rfloor$个结点。所以在一个高度为h的结点上允许MAX-HEAPIFY的代价为O(h)。所以BUILD-MAX-HEAP的总代价为}
	
	\noindent{$\sum_{h=0}^{\lfloor lgn\rfloor}\lceil\frac{n}{2^{h+1}}\rceil O(h)=O(n\sum_{h=0}^{\lceil lgn\rceil}\frac{h}{2^{h}})=O(n)$}
	\section*{2.}
	\noindent{(a)}
	\par{设随机数组有n个数,分别为A[0]A[1]...A[n-1],设有X个数被划分在左半部分,Y个数被划分在右半部分,X+Y=n。$\left|X-Y\right|$的值越大,划分越不平衡。}
	\par\noindent{分三种情况讨论:}\\
	 1. X$\le\alpha$n时,Y$\ge(1-\alpha)n$,$\left|X-Y\right|\textgreater(1-2\alpha)n\textgreater0$\\
	 2. $\alpha n\textless X\le(1-\alpha)$n时,$(1-\alpha)n\le Y\textless\alpha n$,$0\le\left|X-Y\right|\textless(1-2\alpha)n$\\
	 3. $X\textgreater(1-\alpha)n$时,Y$\textless\alpha n$,$\left|X-Y\right|\textgreater(1-2\alpha)n$\\
	 显然情况2最为趋近平衡。X取值在$\alpha n$和$(1-\alpha)n$之间更加平衡。由于X落在1到n之间的概率是等可能的,服从均匀分布。所以X落在$\alpha n$和(1-$\alpha n$)之间的概率为1-2$\alpha$,即PARTITION产生更平衡的划分的概率为1-2$\alpha$。
	\\
	\\
	\noindent{(b)}
	\par{对于叶结点的最小深度:}
	\par{由于$\alpha\le\frac{1}{2}$,所以最小深度的结点应该在$\alpha$划分中。设x为深度:}
	\par{则$n\alpha^{x}=1$,取对数$x+log_{\alpha}n=0$,即$x+\frac{lgn}{lg\alpha}=0$所以最小深度为$-\frac{lgn}{lg\alpha}$}
	\par{对于叶结点的最大深度:}
	\par{由于$\alpha\le\frac{1}{2}$,所以最大深度的结点应该在$1-\alpha$划分中。设x为深度:}
	\par{则$n(1-\alpha)^{x}=1$,取对数$x+log_{1-\alpha}n=0$,即$x+\frac{lgn}{lg1-\alpha}=0$所以最大深度为$-\frac{lgn}{lg1-\alpha}$}
\end{document}