\documentclass{article}
\usepackage{graphicx}
\usepackage{fontspec}
\setmainfont{Microsoft YaHei}
\usepackage{geometry}
\usepackage{ctex}
\usepackage{algorithm}  
\usepackage{algorithmicx}  
\usepackage{algpseudocode}  
\usepackage{amsmath}  
\title{HW7}
\author{王嵘晟 \quad PB1711614}
\date{}
\begin{document}
	\maketitle
	\section*{1.}
	\par{最长简单路径从点s开始,一定经过s的某条边,且在到达t前不会再次路过已经经过的点。所以状态转移方程为:}
	$$LogestPath(G,s,t)=MAX(LogestPath(G',s',t))+1,G'=G-\{s\},ss'\in E$$
	\par{当s=t时,最长简单路径长度为0,当S$\not=$t时,根据状态转移方程由广度优先搜索可以寻找到最长的简单路径。由于这样需要遍历整张图才能找到最长简单路径,所以时间复杂度为$O(|V|^{2}2^{|V|})$。}
	\section*{3.}
	\par{由于员工及其直接主管不同时出席,所以如果树根r,即主席包含在一个最优解中,则其子辈不在最优解中,孙辈一定在最优解中。只需求解根在r孙辈的最优子问题。同样如果树根不在最优解中,则其子辈一定在最优解中。只需求解根在r子辈的最优子问题。所以首先建立由顶点索引的表A,将根在该结点上的所有子树中员工的“宴会交际能力”由大到小排序。然后建立另一个表B,B[n]表示顶点n上的子树的宴会交集评分之和最大的宾客名单。左孩子右兄弟树为T。对于每个叶子结点L,如果L的交际能力$\textgreater$0,则B[L]={L},A[L]=L.ability。否则B[L]={$\emptyset$},A[L]=0。通过迭代可以求解子问题所在结点的父结点的子问题,对结点x,有转移方程:}
	$$A[x]=MIN(\Sigma_{y\ is\ x's\ child}A[y],\Sigma_{y\ is\ x's\ grandchild}A[y]) $$
	\par{当n为员工数量时,时间复杂度$O(n^{2})$。}
	\begin{algorithm} 
	 \caption*{2. 动态规划求解0-1背包}
	 \begin{algorithmic}[1] 
	 	\Function {Dynamic-0-1-Pack}{$v,w,n,W$}  
	 	\For{$w\quad from\quad 0\quad to\quad W$}
	 		\State $c[0,w]=0$
	 	\EndFor
	 	\For{$i\quad from\quad 1\quad to\quad n$}
	 		\State $c[i,0]=0$
	 		\For{$w\quad from\quad 1\quad to\quad W$}
	 			\If{$w_{i}\le W$}
	 				\If{$v_{i}+c[i-1,w-w_{i}]\le c[i-1,w]$} 
	 					\State $c[i,w]=v_{i}+c[i-1,w-w_{i}]$
	 				\Else
	 					\State $c[i,w]=c[i-1,w]$
	 				\EndIf
	 			\Else
	 				\State $c[i,w]=c[i-1,w]$
	 			\EndIf
	 		\EndFor
	 	\EndFor
	 	\State\Return $c[n,w]$
	 	\EndFunction
	 \end{algorithmic}  
	\end{algorithm}
	\section*{4.}
	\par{首先对这个点集使用归并排序有小到大排序,设排序后的点集为${y_{1},y_{2},...y_{n}}$,第一个区间为$[y_{1},y_{1}+1]$,如果$y_{i}$是不包含任何现有的区间内点的最左端点,则下一个区间为$[y_{i},y_{i}+1]$。使用贪心算法,可以保证集合中每个元素必然被包含在一个区间内,所以这样处理一定可以得到最优解。这种算法的时间复杂度为$O(nlgn)$}
	\section*{5.}
	\par{如果要找零的$n\ge 25$,则找零25美分,n=n-25,直到n比25要小,若此时$n\ge 10$,则继续找零10美分,以此类推可以保证找零用硬币最少。不采用这种方法找零,找零的硬币数一定大于等于这种方法,所以这种方法为最优的。}
\end{document}