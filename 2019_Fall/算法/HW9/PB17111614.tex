\documentclass{article}
\usepackage{graphicx}
\usepackage{fontspec}
\setmainfont{Microsoft YaHei}
\usepackage{geometry}
\usepackage{ctex}
\usepackage{algorithm}  
\usepackage{algorithmicx}  
\usepackage{algpseudocode}  
\usepackage{amsmath}  
\title{HW9}
\author{王嵘晟 \quad PB1711614}
\date{}
\begin{document}
	\maketitle
	\section*{1.}
	\subsection*{a.}
	\par{充分性:因为$\forall v\in V$,有$in-degree(v)=out-degree(v)$。所以对于每个结点,出边数等于入边数,则图G中每个结点都是偶次的。所以G是欧拉图,则强连通有向图G=(V,E)中一定有一条欧拉回路。}
	\par{必要性:因为强连通有向图G中有欧拉回路,所以G是欧拉图。则G没有奇次顶。假设$\exists v\in V $,$ in-degree(v)\neq out-degree(v)$则一定$\exists e\in E$,经过e后进入v没有出边或者经过e后离开v到了下一个顶点不能继续到再下一个顶点,这与图G示欧拉图矛盾。所以$in-degree(v)=out-degree(v)$。}
	\subsection*{b.}
	\par{用FE算法遍历图的每一条边来找到欧拉回路,详见算法1}
	\begin{algorithm} 
		\caption*{找图G的欧拉回路}
		\begin{algorithmic}[1] 
			\Function {Find-Euler-Circuit}{$G$}  
			\For{$(v_{i},v_{j})\in E$}	
				\State $Make\ (v_{i},v_{j}) white$
			\EndFor
			\State $Choose\ v_{0}\ to\ be\ the\ source\ vertex,\ insert\ v\ to\ list\ L$
			\While {$edge(v_{0},v_{1})\ comes\ out\ from\ v_{0}\ is\ white$}
				\State $Make\ (v_{0},v_{1})\ black$
				\State $u=v$
				\State $Insert\ v_{1}\ to\ L$
			\EndWhile
			\EndFunction
		\end{algorithmic}  
	\end{algorithm}
	\par{在FE算法中,由于遍历了每一条边,所以算法的时间复杂度为$O(E)$。}
	\section*{2.}
	\subsection*{a.}
	\par{将这个问题转化为图问题,每种货币$c_{i}$构成图的顶集,任意两种货币间的汇率构成图的边集$(c_{i},c_{j})$与$(c_{j},c_{i})$,对每条边$R[i_{1},i_{2}]$取对数并取相反数得$-lnR[i_{1},i_{2}$,以此为边权,将原问题转化为在图中找负权回路问题。使用Bellman Ford算法。算法的运行时间为$O(VE)$,则带入$|V|=k$,时间为$O(\frac{k^{2}(k-1)}{2})$}
	\subsection*{b.}
	\par{最初的操作跟a过程相同,将原问题转化为在图上找负权回路的问题。然后对每个边做松弛操作,共$|V|-1$次。记录每个顶点的d值,然后松弛每个边$|V|$次。然后来检查记录的顶点的d值哪些有所减少,对于d减少的顶点,都位于可能不相交的负权回路上。用集合S来记录这些顶点,为了找到一条回路,对于S中的任意一个顶点,用贪心算法来找出它通过边可达的其他所有顶点,用这种方法可以找到一条负权回路。找到这样一个序列后执行打印操作。由于之前的松弛操作的时间为$O(V)$,执行Bellman Ford算法时间为$O(VE)$,带入得总的时间复杂度为$O(\frac{k^{2}(k-1)}{2})$}
	\section*{3.}
	\par{由引理25.1,$\hat{\omega}(u,v)=\omega(u,v)+h(u)-h(v)$,所以对于环路c来说,重新加权后任意一条环的总权重不变。所以由$\hat{\omega}(c)=\omega(c)=0$,重新加权后的图中,环还是之前的环,总权重为0。由于重新加权后不再存在负权边,可环路的总权重为0,所以每条边的权重$\hat{\omega}(u,v)=0$。}
	\section*{4.}
	\subsection*{a.}
	\par{如果对于边(u,v)不存在最小切割,则最大流没有办法增加,则剩余的网络中不存在增广路径。所以当增广路径存在时,执行一次Ford-Fullkerson迭代,找到并增加该路径,由于边的容量是整数,所以每次增加都是整数。由于每条边的流严格增加,并且每次增加一个整数。所以Ford-Fullkerson第三行的while循环每次迭代使得流量+1,直到达到最大流。所以为了找到增广路径,使用BFS,用的时间为$O(V+E)$。}
	\subsection*{b.}
	\par{如果边的流量比容量小至少1,则不会有任何变化。否则使用BFS在$O(V+E)$的时间内查找从s到t包含边(u,v)的路径,将该路径上每个边的流减小1,然后在$O(V+E)$的时间内运行Ford-Fullkerson算法循环的迭代,由于每次增加的都是整数值,所以最终结果要么是找不到增广路径,要么是最后总的最大流减少1然后结束。}
\end{document}