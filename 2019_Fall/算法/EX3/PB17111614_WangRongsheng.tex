\documentclass[11pt,a4paper]{article}
\usepackage{graphicx}
\usepackage{fontspec}
\setmainfont{Microsoft YaHei}
\usepackage{geometry}
\usepackage{ctex}
\usepackage{algorithm}  
\usepackage{algorithmicx}  
\usepackage{algpseudocode}  
\usepackage{amsmath}  
\title{EX3 实验报告}
\author{王嵘晟 \quad PB1711614}
\date{}
\begin{document}
	\maketitle
	\section*{EX3-1}
	\par{思路:这道题我对于输入的点坐标,先进行对横坐标由小到大排序,使用计数排序。然后对排序好的点,此题可以转化为类似最长上升子串的动态规划经典模型。使用自底向上法做动态规划,用MaxLen[n]记录前n个点中最长的上升子串长度。转移方程:}
	$$ MaxLen[n]=\left\{
	\begin{aligned}
	1\quad ,\quad n=1 \\
	MAX_{1\le i\textless n, A[i]\textless A[n]}(MaxLen[i]+1,MaxLen[n])\quad ,\quad n>1
	\end{aligned}
	\right.
	$$
	\par{计数排序时间复杂度$O(n)$,对1到n确定MaxLen的值由于双重循环时间复杂度$O(n^{2})$,遍历MaxLen找到需要输出的最长子串值时间复杂度$O(n)$,所以总的时间复杂度为$O(n^{2})$}
	\section*{EX3-2}
	\par{思路:这道题我开设了二维数组,m[i][j]表示从第i个数组到第j个数组合并所需要的最小开销,状态转移方程:}
	$$ m[i][j]=MIN(m[i][j],m[i][k]+m[k+1][j]+SUM(a,i,j+1)),i\le k\le j$$
	\par{由于这里使用了三重循环,第一重循环k为数组序列长度,i为起始数组号,k为序列中间指针,所以时间复杂度为$O(n^{3})$}
	\section*{EX3-3}
	\par{思路:我开了个结构体来存储每一种物品的质量,价值和可选数量。多重背包我一开始想转化为部分背包来处理,使用贪心算法,结果发现了自己逻辑有错误,贪心算法并不能行得通。于是尝试最普通的三重循环做动态规划,结果发现会超时。于是采用了二进制优化。对于任何数,一定可以找到一串2的幂次数的组合来表示。所以我用cnt来表示$2^{n},n=0,1,2,3...$,对于第二重循环j从背包最大承重W到每个物体的质量$w_{i}$,j每次减去的量为$min(cnt.num_{i})*w_{i}$。令f[j]表示背包中有最大序号为i的物品时最大价值。转移方程为:}
	$$f[j]=MAX(f[j],f[j-min(cnt.num_{i})*w_{i}]+min(cnt.num_{i})*v_{i}),1\le i\le j\le n$$
	\par{时间复杂度为$O(nlgW)$}
	\section*{EX3-EX}
	\par{思路:这道题我用了一个比较巧妙的方法,把输入的0和1恰好取反,即0变1,1变0,这样就可以用二维数组本身的元素进行计数,把0 1串读入二维数组a,计数数组Square先做第一次计数————边长为1的正方形个数,然后根据状态方程:}
	$$Square[i][j]=\left\{
	\begin{aligned}
	0\quad ,\quad a[i][j]=0 \\
	MIN(Square[i-1][j-1],Square[i-1][j],Square[i][j-1])+1\quad ,\quad a[i][j]=1
	\end{aligned}
	\right.$$
	\par{求出数组Square[i][j]的每个元素大小后,做累加即为总的正方形个数,时间复杂度$O(n^{2})$}
\end{document}