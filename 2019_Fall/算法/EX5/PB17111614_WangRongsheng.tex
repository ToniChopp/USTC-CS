\documentclass[11pt,a4paper]{article}
\usepackage{graphicx}
\usepackage{fontspec}
\setmainfont{Microsoft YaHei}
\usepackage{geometry}
\usepackage{ctex}
\usepackage{algorithm}  
\usepackage{algorithmicx}  
\usepackage{algpseudocode}  
\usepackage{amsmath}  
\title{EX5 实验报告}
\author{王嵘晟 \quad PB1711614}
\date{}
\begin{document}
	\maketitle
	\section*{EX5-1}
	\par{道路规划:这道题显然是要寻找一个图的最小生成树,因此我选择使用Kruskal算法,先对每条边的权值进行从小到大排序,然后贪心得使用并查集不相交森林来求解最小生成树,每次加入不形成环的最小权值边。排序使用快排,时间复杂度为$O(MlgM)$,而之后找边的过程时间复杂度为$O(M)$。}
	\par{所以总的时间复杂度为$O(MlgM)$。}
	\section*{EX5-2}
	\par{逃离迷宫:这道题求解从一个顶点到另一个顶点的最短路径,所以是图上最短路径问题。使用Dijkstra算法,并且用二叉堆的数据结构来做到Extract-Min。由于输入有重边,因此我首先用二叉树对输入边进行处理,去掉重边。所用时间为$O(MlgM)$,之后将边存入邻接链表中,并压入最小二叉堆,然后使用Dijkstra算法每次取出堆顶元素,加入到最短路径中。}
	\par{最后总的时间复杂度为$O((M+N)lgM)$。}
	\section*{EX5-3}
	\par{货物运输:这道题是在考察最大流算法,求一个有向图中给定源和汇的最大流。使用Dinic算法,即将给定的图分层,找增广路径时只找从当前层到下一层的增广路径,这样可以大大优化执行时间。用BFS来对图进行分层,用DFS找增广路径。}
	\par{时间复杂度为$O(N^{2}M)$。}
	\section*{EX5-EX}
	\par{图中的最大集合:这道题实际上是求一个有向图中的最大团,具体思路为先用Tarjan算法来缩点,然后对于缩点后的原图,重建网络图,这样就可以用DFS和动态规划的方法来求解最大团了。转移方程为:$$dp[u] = max(dp[u], num[u]+dfs(v));$$}
	\par{时间复杂度为$O(M)$。}
\end{document}